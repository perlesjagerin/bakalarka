%%%  Ukázkový text a dokumentace stylu pro text závěrečné (bakalářské a
%%%  diplomové) práce na KI PřF UP v Olomouci
%%%  Copyright (C) 2012 Martin Rotter, <rotter.martinos@gmail.com>
%%%  Copyright (C) 2014 Jan Outrata, <jan.outrata@upol.cz>


%%  Pro získání PDF souboru dokumentu je třeba tento zdrojový text v
%%  LaTeXu přeložit (dvakrát) programem pdfLaTeX.

%%  V případě použití programu BibLaTeX pro tvorbu seznamu literatury
%%  je poté ještě třeba spustit program Biber s parametrem jméno
%%  souboru zdrojového textu bez přípony a následně opět (dvakrát)
%%  přeložit zdrojový text programem pdfLaTeX.

%%  Postup získání Postscriptového souboru je popsán v dokumentaci.


%%  Třída dokumentu implementující styl pro závěrečnou práci. Vybrané
%%  nepovinné parametry (ostatní v dokumentaci):

%%  'master' pro sazbu diplomové práce, jinak se sází bakalářská práce

%%  'program=kód' pro Váš studijní program/obor (specializaci), kódy
%%  pro diplomovou práci 'infoi' pro Informatiku (Obecná informatika),
%%  'infui' pro Informatiku (Umělá inteligence), 'ainfpst' pro
%%  Aplikovanou informatiku (Počítačové systémy a technologie), 'uinf'
%%  pro Učitelství informatiky pro střední školy, 'binf' pro
%%  Bioinformatiku, 'inf' pro Informatiku (bez specializací) a 'ainf'
%%  pro Aplikovanou informatiku (bez specializací), jinak je výchozí
%%  ainfvs pro Aplikovanou informatiku (Vývoj software), a pro
%%  bakalářskou práci 'infoi' pro Informatiku (Obecná informatika),
%%  'itp' pro Informační technologie v prezenční formě, 'itk' pro
%%  Informační technologie v kombinované formě, 'infv' pro Informatiku
%%  pro vzdělávání, 'binf' pro Bioinfomatiku, 'inf' pro Informatiku
%%  (bez specializací), 'ainfp' pro Aplikovanou informatiku (bez
%%  specializací) v prezenční formě, 'ainfk' pro Aplikovanou
%%  informatiku (bez specializací) v kombinované formě, jinak je
%%  výchozí infpvs pro Informatiku (Programování a vývoj software)

%%  'printversion' pro sazbu verze pro tisk (nebarevné logo a odkazy,
%%  odkazy s uvedením adresy za odkazem, ne odkazy do rejstříku),
%%  jinak verze pro prohlížeč

%%  'biblatex' pro zapnutí podpory pro sazbu bibliografie pomocí
%%  BibLaTeXu, jinak je výchozí sazba v prostředí thebibliography

%%  'language=jazyk' pro jazyk práce, jazyky english pro anglický,
%%  slovak pro slovenský, jinak je výchozí czech pro český

%%  'font=sans' pro bezpatkový font (Iwona Light), jinak je výchozí
%%  serif pro patkový (Latin Modern)

%%  'figures, tables, theorems a sourcecodes' pro sazbu seznamu
%%  obrázků, tabulek, vět a zdrojových kódů, jinak při =false se
%%  nesází (u theorems a sourcecodes výchozí)

\documentclass[
program=itk,
%  printversion,
  biblatex,
%  language=english,
%  font=sans,
  figures=false,
%  tables=false,
%  theorems,
%  sourcecodes,
  glossaries,
  index
]{kidiplom}

%% Informace pro úvodní strany. V jazyku práce (pokud není v komentáři
%% uvedeno česky) a anglicky. Uveďte všechny, u kterých není v
%% komentáři uvedeno, že jsou volitelné. Při neuvedení se použijí
%% výchozí texty. Text pro jiný než nastavený jazyk práce (nepovinným
%% parametrem language makra \documentclass, výchozí český) se zadává
%% použitím makra s uvedením jazyka jako nepovinného parametru.

%% Název práce, česky a anglicky. Měl by se vysázet na jeden řádek.
\title{Interaktivní webová platforma pro rezervaci vstupenek na studentské akce}
\title[english]{Interactive web platform for ticket reservation on student events}

%% Volitelný podnázev práce, česky a anglicky. Měl by se vysázet na
%% jeden řádek. Výchozí je prázdný.
%% \subtitle{Ukázkový text a dokumentace stylu v \LaTeX{}u}
%% \subtitle[english]{Sample text and documentation of the \LaTeX{} style}

%% Jméno autora práce. Makro nemá nepovinný parametr pro uvedení
%% jazyka.
\author{Tuan Anh Nguyen}

%% Jméno vedoucího práce (včetně titulů). Makro nemá nepovinný
%% parametr pro uvedení jazyka.
\supervisor{Mgr. Jiří Zacpal, Ph.D.}

%% Volitelný rok odevzdání práce. Výchozí je aktuální (kalendářní)
%% rok. Makro nemá nepovinný parametr pro uvedení jazyka.
%\yearofsubmit{\the\year}

%% Anotace práce, včetně anglické (obvykle překlad z jazyka
%% práce). Jeden odstavec!
\annotation{Ukázkový text závěrečné práce na Katedře informatiky
  Přírodovědecké fakulty Univerzity Palackého v Olomouci, který je
  zároveň dokumentací stylu pro text práce v \LaTeX{}u. Zdrojový text
  v \LaTeX{}u je doporučeno použít jako šablonu pro text skutečné
  závěrečné práce studenta.}

\annotation[english]{Sample text of thesis at the \kitextdepten,
  \kitextfacultyen, \kitextuniven{} and, at the same time,
  documentation of the \LaTeX{} style for the text. The source text in
  \LaTeX{} is recommended to be used as a template for real student's
  thesis text.}

%% Klíčová slova práce, včetně anglických. Oddělená (obvykle) středníkem.
\keywords{styl textu; závěrečná práce; dokumentace; ukázkový text}
\keywords[english]{text style; thesis; documentation; sample text}

%% Volitelná specifikace příloh textu práce, i anglicky. Výchozí je
%% 'elektronická data v systému katedry informatiky / electronic data
%% in system of department of computer science'.
%\supplements{nejlepší software všech dob}
%\supplements[english]{the best software of all times}

%% Volitelné poděkování. Stručné! Výchozí je prázdné. Makro nemá
%% nepovinný parametr pro uvedení jazyka.
\thanks{Děkuji, děkuji, děkuji.}

%% Cesta k souboru s bibliografií pro její sazbu pomocí BibLaTeXu
%% (zvolenou nepovinným parametrem biblatex makra
%% \documentclass). Použijte pouze při této sazbě, ne při (výchozí)
%% sazbě v prostředí thebibliography.
\bibliography{bibliografie.bib}

%% Další dodatečné styly (balíky) potřebné pro sazbu vlastního textu
%% práce.
\usepackage{lipsum}
\usepackage{longtable}

\begin{document}
%% Sazba úvodních stran -- titulní, s bibliografickými údaji, s
%% anotací a klíčovými slovy, s poděkováním a prohlášením, s obsahem a
%% se seznamy obrázků, tabulek, vět a zdrojových kódů (pokud jejich
%% sazba není vypnutá).
\maketitle

%% Vlastní text závěrečné práce. Pro povinné závěry, před přílohami,
%% použijte prostředí kiconclusions. Povinná je i příloha s obsahem
%% elektronických dat.

%% -------------------------------------------------------------------

\newcommand{\BibLaTeX}{\textsc{Bib}\LaTeX}

% \noindent\textcolor{red}{\LARGE Upozornění: Následující text
%   dokumentace stylu, vyjma přílohy~\ref{sec:ObsahData}, je rozpracovaná
%   a (značně) neúplná verze!!!}

\section{Průzkum existujících řešení}
Pro zkoumání budou vybrány následující platformy:
\begin{itemize}
    \item Eventbrite
    \item Ticketmaster
    \item Meetup
    \item Eventzilla
\end{itemize}

\subsection{Popis jednotlivých řešení}

\subsubsection{Eventbrite}
je platforma využívající technologie Ruby on Rails, React a PostgreSQL. Mezi hlavní funkce patří vytváření, modifikace a rušení akcí, rezervace a nákup vstupenek online, podpora různých typů vstupenek (volný vstup, placené vstupenky), integrovaný platební systém a možnost refundací. Výhodou Eventbrite je jeho vysoký výkon a škálovatelnost, stejně jako podpora různých platebních bran. Na druhou stranu se jedná o řešení s vyššími náklady na implementaci a údržbu a složitým kódem

\subsubsection{Ticketmaster}
je postaven na technologiích Java, Spring Boot, Angular a MySQL. Poskytuje funkce správy a propagace akcí, rezervace a prodeje vstupenek, podporu mobilních vstupenek a systém pro správu vstupenek, včetně reklamací a refundací. Ticketmaster je robustní a zabezpečená platforma, která podporuje velký počet uživatelů a transakcí. Nevýhodou jsou vyšší složitost vývoje a údržby a vysoké náklady na provoz.

\subsubsection{Meetup}
využívá technologie Node.js, Express.js, React a MongoDB. Umožňuje organizaci a správu událostí, potvrzení účasti (RSVP) a podporu plateb za akce. Mezi jeho výhody patří flexibilní a rychlý vývoj díky použitým technologiím a dobrá podpora komunitních funkcí. Omezení spočívají ve škálovatelnosti pro velmi velké akce a méně pokročilých funkcích pro správu plateb.

\subsubsection{Eventzilla}
je platforma postavená na PHP, Laravel, Vue.js a PostgreSQL. Nabízí funkce pro vytváření a správu událostí, rezervaci a prodej vstupenek, podporu různých typů vstupenek a integraci s různými platebními bránami. Výhodou Eventzilla je snadné přizpůsobení a rozšíření funkcí, stejně jako dobrá podpora různých platebních systémů. Nevýhodou je méně známá značka a omezené analytické nástroje.

\subsection{Hodnocení kladů a záporů}
Pokud se podíváme na hodnocení kladů a záporů jednotlivých řešení, Eventbrite je vysoce výkonné a škálovatelné, ale s vyššími náklady a komplexností. Ticketmaster je robustní a zabezpečené, ale drahé a složité. Meetup nabízí flexibilní a rychlý vývoj, ale je omezené pro velké akce. Eventzilla je přizpůsobitelné a cenově dostupné, ale méně známé a s omezenými analytickými nástroji.

\subsection{Vyhodnocení přínosů}
Co se týče přínosů, Eventbrite inspiruje k vytvoření výkonného a škálovatelného řešení. Ticketmaster ukazuje, jak vybudovat robustní a zabezpečenou platformu. Meetup klade důraz na komunitní funkce a rychlý vývoj. Eventzilla nabízí přístup k přizpůsobitelnému a cenově dostupnému řešení.

\subsection{Přínosy vlastní práce}
Podpora studentských aktivit bude posílena vytvořením platformy pro rezervaci vstupenek, což usnadní organizaci a propagaci studentských akcí. To přispěje k aktivnímu studentskému životu a komunitě. Zjednodušení správy akcí prostřednictvím automatizace a zjednodušení procesů spojených s organizací akcí ulehčí práci studentským organizátorům, kteří se tak mohou více soustředit na kreativní a obsahovou část akcí.

Zvýšení účasti na akcích je dalším přínosem, neboť snadná dostupnost a rezervace vstupenek může vést ke zvýšenému zapojení studentů do komunitního života. Bezpečnost a pohodlí budou zajištěny implementací moderních platebních systémů a ochranou osobních údajů podle GDPR, což zvýší důvěru uživatelů v používání platformy.

Rozvoj digitálních dovedností u studentů zapojených do vývoje a správy platformy je další významný přínos. Studenti získají praktické zkušenosti s moderními technologiemi a vývojem webových aplikací, což je cenné pro jejich budoucí profesní kariéru. Ekologický přínos se projeví digitalizací vstupenek a snížením potřeby fyzických vstupenek, což přispěje k ochraně životního prostředí.

Zahájení činnosti studentské organizace zaměřené na organizaci a propagaci studentských akcí může být dalším přínosem této práce. Tím se obohatí univerzitní komunita a nabídnou se nové příležitosti pro zapojení studentů.

\section{Specifikace řešení}

\subsection{Uživatelské role}
Platforma bude disponovat několika uživatelskými rolemi, z nichž každá bude mít specifické přístupy a oprávnění. 

\textbf{Administrátor} bude mít plný přístup ke všem funkcím systému, což mu umožní kontrolovat a spravovat všechny aspekty platformy. 

\textbf{Organizátor} bude mít přístup k vytváření a správě vlastních akcí, což zahrnuje možnosti přidávat nové události, upravovat stávající a rušit je podle potřeby. 

Běžný \textbf{uživatel} bude mít přístup k prohlížení nabízených akcí a rezervaci vstupenek, čímž se usnadní zapojení do studentských aktivit.

\subsection{Správa událostí}
V rámci správy událostí bude organizátor moci přidávat nové akce vyplněním detailního formuláře, který bude obsahovat všechny potřebné specifikace události, jako je datum, čas, místo konání a popis. 

Po zveřejnění bude mít možnost upravovat údaje akce, což je užitečné pro případné změny v programu nebo organizační detaily. 

V případě potřeby zrušení události může organizátor tuto možnost využít a současně informovat všechny registrované účastníky o změně nebo zrušení prostřednictvím automatizovaných oznámení.

\subsection{Rezervace}
Uživatelé budou moci prohlížet dostupné akce prostřednictvím přehledného seznamu, který nabídne různé možnosti filtrování a vyhledávání podle specifických kritérií, jako jsou datum, kategorie nebo místo konání. 

Při rezervaci vstupenek si uživatel vybere požadovaný počet vstupenek a vyplní nezbytné osobní údaje. 

Pokud by uživatel potřeboval změnit nebo zrušit svou rezervaci, může to snadno provést přes svůj uživatelský účet, což poskytuje vysokou míru flexibility a pohodlí.

\subsection{Reklamace a vrácení vstupenek}
Pro případ reklamace bude k dispozici uživatelský formulář, který umožní uživatelům podat reklamaci na vstupenky nebo služby spojené s akcí. Tento proces bude zahrnovat postup schválení, který zajistí, že všechny reklamace budou důkladně posouzeny a vyřešeny spravedlivě. 

Vrácení peněz bude řešeno prostřednictvím automatizovaného systému, který zajistí rychlou a efektivní refundaci podle stanovených pravidel, což zvýší důvěru uživatelů v bezpečnost a spolehlivost platformy.

\subsection{Návrh vhodných technologií}

\subsubsection{Backend}
Pro backend je vhodnou volbou \textbf{Node.js}, který se vyznačuje rychlým a efektivním asynchronním zpracováním. Tato vlastnost je klíčová pro aplikace, které musí zvládat vysokou zátěž na I/O operace, jako jsou rezervace a platby. Další výhodou je široká komunita a množství dostupných knihoven, které usnadňují rychlý vývoj a nasazení aplikace.

V kombinaci s Node.js je ideální použít \textbf{Express.js}, což je jednoduchý a minimalistický framework, který umožňuje rychlý vývoj RESTful API. Tento framework je dobře podporován v komunitě Node.js a nabízí velkou flexibilitu při tvorbě serverových aplikací.

\subsubsection{Databáze}
Pro ukládání dat je vhodná \textbf{PostgreSQL}, což je spolehlivá a výkonná relační databáze. Podporuje pokročilé funkce jako je transakční zpracování, což je důležité pro zajištění integrity dat při rezervacích a platbách. PostgreSQL je také dobře škálovatelná a podporuje komplexní dotazy, které mohou být potřebné pro analýzu dat.

\subsubsection{Frontend}
Na frontendu je vhodné použít \textbf{React}, což je vysoce výkonný a flexibilní frontendový framework. React umožňuje vytváření dynamických uživatelských rozhraní a díky komponentovému přístupu usnadňuje správu a opakované použití kódu.

Pro stylování je výhodné využít \textbf{Tailwind CSS}, což je utility-first CSS framework. Tento framework umožňuje rychlé a efektivní stylování komponent a je vysoce přizpůsobitelný. Tailwind CSS také usnadňuje tvorbu responzivních a moderních uživatelských rozhraní.

\subsubsection{Autentizace a autorizace}
Pro autentizaci a autorizaci uživatelů je vhodné použít \textbf{JWT} (JSON Web Tokens). JWT poskytuje bezpečný a škálovatelný způsob autentizace a autorizace uživatelů. Umožňuje snadnou integraci s RESTful API a je vhodný pro aplikace s vysokou úrovní zabezpečení.

\subsubsection{Platby}
Pro zpracování plateb je ideální \textbf{Stripe}, což je flexibilní a spolehlivá platební brána. Stripe podporuje různé platební metody a měny a nabízí robustní API pro integraci plateb do webových aplikací. Zajišťuje vysokou úroveň zabezpečení platebních transakcí.

\subsubsection{Hosting a nasazení}
Pro hosting a nasazení aplikace je vhodná platforma \textbf{Heroku}, která je jednoduchá a uživatelsky přívětivá platforma, která podporuje různé programovací jazyky a frameworky. Nabízí snadnou správu a škálovatelnost aplikací, což je klíčové pro rychlé nasazení a provozování aplikace.

\subsection{Diagram případu použití}
\begin{figure}
    \centering
    \includegraphics[width=1\linewidth]{kidiplom//graphics/BAK1 - Use case diagram.png}
    \caption{Diagram případu užití}
    \label{fig:enter-label}
\end{figure}


\end{document}
